\documentclass[12pt,a4paper,UTF8]{ctexart}
\usepackage{geometry}
	\geometry{left=2.5cm,right=2.5cm,top=0cm,bottom=1cm}
\usepackage{xeCJK,amsmath,paralist,enumitem,booktabs,multirow,graphicx,subfig,setspace}
	\setlength{\parindent}{2em}
\usepackage[colorlinks,linkcolor=blue,urlcolor=blue]{hyperref}

%%%%%%%%%%%%%%%%%%%%%%%%%正文开始%%%%%%%%%%%%%%%%%%%%%%%%%%
\begin{document}

\title{\Large\bfseries Measure the Absorption spectra using fiber spectrometer\footnotemark[1]}
\author{\large\textit{Ziwei Huang}$^{1}$\footnotemark[2],\ \large\textit{Ruijie Huang}$^{2}$\footnotemark[3] \\ 
\small{1,2 Zhongshan School of Medicine, Sun Yat-sen University, Guangzhou  { \rm 510275}, China}}
\date{}
\maketitle\thispagestyle{empty} 

\vspace{-1.5em}
\begin{spacing}{1.5}
{\bfseries Abstract:}
Bill-Lambert's law states that when light passes through a solution of a certain concentration, the light-absorbing molecules in the solution will absorb the light, resulting in a weakening of the light intensity. 
At low concentrations, the absorption coefficient of the solution is proportional to the concentration. 
This theory reveals the law of light absorption in solution and has important applications in fields such as the rapid measurement of solution concentrations using absorption spectroscopy. 
In this experiment, we use a fiber optic spectrometer to measure the absorption spectra of the light generated by a bromine-tungsten lamp passing through different concentrations of red ink solutions 
and calculate the absorption coefficients to verify Beer's law and its conditions of validity.

We first measured and compared the spectra of the bromine-tungsten lamp, the absorption spectra of the empty cuvette and pure water, and found that pure water generally absorbs light in the visible range with almost constant peak spectral wavelength, while the absorption spectra of the empty cuvette is relatively weak in intensity. 
Next, we measured the absorption spectra of different concentrations of red ink solutions and found that the peak spectral wavelength of the absorption spectrum gradually increased as the concentration of red ink increased. 
Finally, we calculated the absorption coefficients of different concentrations of red ink solutions and found that at lower concentrations, the absorption coefficients were proportional to the concentration, while at higher concentrations, the absorption of light by the solution tended to saturate.
\par
\bfseries{Key words}: Fiber optic spectrometer, Absorption spectra, Bill-Lambert's law
\vspace{2em}
\end{spacing}

\renewcommand{\thefootnote}{\fnsymbol{footnote}}
\footnotetext[1]{{Supported and taught by Luyoutang, School of Physics, Sun Yat-sen University}}
\footnotetext[2]{{Corresponding author. ID:\ 20980066 E-mail:\ \url{huangzw29@mail2.sysu.edu.cn}}}
\footnotetext[3]{{Participant. ID:\ 20980062}}
\end{document}