\documentclass[12pt,a4paper,UTF8]{ctexart}
\usepackage{geometry}
	\geometry{left=2.5cm,right=2.5cm,top=0cm,bottom=1cm}
\usepackage{xeCJK,amsmath,paralist,enumitem,booktabs,multirow,graphicx,subfig,setspace}
	\setlength{\parindent}{2em}
\usepackage[colorlinks,linkcolor=blue,urlcolor=blue]{hyperref}

%%%%%%%%%%%%%%%%%%%%%%%%%正文开始%%%%%%%%%%%%%%%%%%%%%%%%%%
\begin{document}

\title{\Large\bfseries Measure the Absorption spectra using fiber spectrometer\footnotemark[1]}
\author{\large\textit{Ziwei Huang}$^{1}$\footnotemark[2],\ \large\textit{Ruijie Huang}$^{2}$\footnotemark[3] \\ 
\small{1,2 Zhongshan School of Medicine, Sun Yat-sen University, Guangzhou  { \rm 510275}, China}}
\date{}
\maketitle\thispagestyle{empty} 

\vspace{-1.5em}
\begin{spacing}{1.5}
{\bfseries Abstract:}
Fraunhofer single-slit diffraction describes a phenomenon in which when parallel light passes through a slit with dimensions close to the wavelength, it deviates from the direction of propagation and forms diffraction streaks at infinity. 
The diffraction streak has a primary maximum with a great energy concentration, and those secondary maximums symmetrically distributed around it. 
Also, the light intensity decays rapidly as the diffraction order increases. 
Near the center of the streak, the angular width of the secondary streak is approximately equal to the half-angle width of the primary maximum. 
In this experiment, we use an optical power meter to measure the relative intensity distribution of the streaks of He-Ne laser Fronhofer single-slit diffraction pattern and compare it with the theoretical curve. 
We also check whether the optical power meter works in its linear region. 
Finally, we conducted a simulation experiment on the $Seelight$ optical simulation platform.

First, we measured the relative intensity distribution of the stripes at each level of He-Ne laser Fronhofer single-slit diffraction pattern and compared it with the theoretical curve. 
We found that the primary maximum is in good agreement with the theoretical curve, but those secondary maximums are relatively low. What’s more, the intensity at those dark stripes is not zero. 
We also calculated the  distance of the dark streaks at each level, and found that it was in good agreement with the theoretical prediction. 
Next, we measured the optical power of the bromine tungsten light at different distances using an optical power meter, and confirmed that the optical power meter operated within the linear region. 
Finally, we built an model on the $Seelight$ optical platform using the same experimental parameters, conducted simulation experiments and verified the simulation results.
\par
\bfseries{Key words}: Fiber optic spectrometer, Absorption spectra, Bill-Lambert's law
\vspace{2em}
\end{spacing}

\renewcommand{\thefootnote}{\fnsymbol{footnote}}
\footnotetext[1]{{Supported and taught by Luyoutang, School of Physics, Sun Yat-sen University}}
\footnotetext[2]{{Corresponding author. ID:\ 20980066 E-mail:\ \url{huangzw29@mail2.sysu.edu.cn}}}
\footnotetext[3]{{Participant. ID:\ 20980062}}
\end{document}