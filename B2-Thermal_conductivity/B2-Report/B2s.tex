% %---------------------导言区---------------------------%
\documentclass[12pt,a4paper,UTF8]{ctexart}
\usepackage{geometry}
	\geometry{left=2.5cm,right=2.5cm,top=3.2cm,bottom=2.8cm}
\usepackage{xeCJK,amsmath,paralist,enumitem,booktabs,multirow,graphicx,subfig,setspace,listings,lastpage}
\usepackage[colorlinks,
            linkcolor=blue,       
            anchorcolor=blue,  
            citecolor=blue,       
            ]{hyperref}
	\setlength{\parindent}{2em}
	\lstset{language=Python}
\usepackage{fancyhdr}
	\pagestyle{fancy}
	\rhead{B2 准稳态法测量不良导体的热导率}
	\lhead{基础物理实验\uppercase\expandafter{\romannumeral2}补充材料}
	\cfoot{Page \thepage/\pageref{LastPage}}  %当前页\总页数
	\rfoot{\today}
	\renewcommand{\headrulewidth}{0.4pt}
	\renewcommand{\theenumi}{(\arabic{enumi})}

\renewcommand{\thefigure}{S\arabic{figure}}
\renewcommand{\thetable}{S\arabic{table}}
%%%%%%%%%%%%%%%%%%%%%%%%%%%%%%%%%%%%%%%%%%%%%%%%%%%%%%%%%%
%%%%%%%%%%%%%%%%%%%%%%%%%正文开始%%%%%%%%%%%%%%%%%%%%%%%%%%
%%%%%%%%%%%%%%%%%%%%%%%%%%%%%%%%%%%%%%%%%%%%%%%%%%%%%%%%%%

\begin{document}

\begin{center}
\LARGE\textbf{B2 准稳态法测量不良导体的热导率 补充材料}
\end{center}

%%信息
\begin{doublespacing}
	\centering
	\begin{tabular}{ll}
	 & \\
	{\CJKfontspec{Droid Sans Fallback} 实验人:黄子维 20980066} & {\CJKfontspec{Droid Sans Fallback}合作者:黄睿杰 20980062}\\
	{\CJKfontspec{Droid Sans Fallback} 实验时间:2021.11.25~星期四~上午} & {\CJKfontspec{Droid Sans Fallback} 室温:22$^{\circ}$C~相对湿度:40\%}
	\end{tabular}
\end{doublespacing}


\subsection*{实验设备}
\begin{table}[htbp]
    \centering
        \begin{tabular}{cccc}
            \toprule
            编号 &名称 &数量 &仪器参数及型号 \\
            \midrule
            1	&比热/导热系数实验仪	&1	&$ZKY-BRDR$    \\    
            2	&样品架和保温杯	&1	&$ZKY-BRDR/S$ \\
            3	&有机玻璃样品	&4	& \\
            4	&橡胶材料样品	&4	&   \\
            5	&T型热电偶	&2	&    \\
            \bottomrule
        \end{tabular}
        \caption{\textbf{实验设备}}
\end{table}	

\subsection*{实验参数}

\begin{table}[htbp]
    \centering
        \begin{tabular}{cc}
            \toprule
            项目 &参数  \\
            \midrule
            加热电压$U$ &$17.0 V$ \\
            电阻$r$ &$110 \Omega$ \\
            有机玻璃密度$\rho$ &$1196 kg/m^3$ \\
            橡胶材料密度$\rho$ &$1374 kg/m^3$ \\
            \bottomrule
        \end{tabular}
        \caption{\textbf{实验参数}}
\end{table}	





\subsection*{思考题}
\subsubsection*{1. 样品导热系数的大小与温度有什么关系?}
样品的导热系数随温度升高而变大。这是因为温度升高会加快分子热运动,促进固体骨架的导热及孔隙内流体(如气体)的对流传热。
此外,孔壁间辐射换热也会随温度的升高而加强。
\subsubsection*{2. 样品导热系数的大小与导热性能有什么关系?}
导热系数越大,导热性能越好。在热流密度和厚度相同时,物体高温侧壁面与低温侧壁面间的温度差随导热系数增大而减小。
\subsubsection*{3. 分析本实验的主要误差来源?}
\begin{enumerate}[label=\arabic*.]
	\item 实验样品材料并非理想无限大导热模型,虽然对面积引入了边缘修正因子$A$,但$A$大小的确定是经验性的,这可能会引入较大误差。
	\item 实验用样品材料尺寸并非绝对一致均匀相同,中心面两侧样品厚度并非完全相等,这可能对导热带来影响。
	\item 实限于实验条件,冷端并非置于摄氏零度环境,但我们参考的热电偶分度表是基于冷端摄氏零度标定的,因此热电偶温度-电压系数可能并不准确。
	\item 橡胶材料具有弹性,使用游标卡尺测量难以控制,这会引入测量误差。
	\item 由于实验须在升温过程中手动记录数据并切换电压测量,难以把握记录时机,可能时间记录并不完全准确。
\end{enumerate}
\subsubsection*{4. 本实验中怎样实现稳定导热?如何判定已经到达稳定导热状态?}
稳定导热:
\begin{enumerate}[label=\arabic*.]
	\item 实验采用薄膜加热器,其加热功率在加热面上均匀且可精确控制。提供稳定电压,则加热功率可以维持恒定。
	\item 使用四块完全相同的待测样品两两夹持薄膜加热器,对称平衡热传导,保证导热稳定。
\end{enumerate}

判定:
\begin{enumerate}[label=\arabic*.]
    \item 中心面温度随时间线性上升
    \item 中心面和导热面温差恒定
\end{enumerate}



\subsection*{项目源码}
\href{https://github.com/Jeg-Vet/SYSU-PHY-EXP/tree/main/B2-Thermal_conductivity}{SYSU-PHY-EXP/B6 Thermal conductivity.Jeg-Vet(github.com)}

\end{document}